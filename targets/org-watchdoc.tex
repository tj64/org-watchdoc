% Created 2014-04-10 Do 01:36
\documentclass[11pt]{article}
\usepackage[utf8]{inputenc}
\usepackage[T1]{fontenc}
\usepackage{fixltx2e}
\usepackage{graphicx}
\usepackage{longtable}
\usepackage{float}
\usepackage{wrapfig}
\usepackage{rotating}
\usepackage[normalem]{ulem}
\usepackage{amsmath}
\usepackage{textcomp}
\usepackage{marvosym}
\usepackage{wasysym}
\usepackage{amssymb}
\usepackage{hyperref}
\tolerance=1000
\author{Thorsten Jolitz\thanks{tjolitz at gmail dot com}}
\date{\textit{<2014-04-09 Mi>}}
\title{Org-watchdoc}
\hypersetup{
 pdfkeywords={org-mode, outshine, outorg, exporter, synchronization},
  pdfsubject={Watchdog for exported Org-mode trees},
  pdfcreator={Emacs 24.3.1 (Org mode 8.2.5h)}}
\begin{document}

\maketitle
\setcounter{tocdepth}{3}
\tableofcontents


\section{org-watchdoc.el --- Watchdog for exported Org-mode trees}
\label{sec-1}
\begin{verbatim}
EXPORT_OPTIONS: prop:nil
wdoc_1992rwM: /home/tj/git/org-watchdoc/README.md /home/tj/git/org-watchdoc/export-templates/org-watchdoc-gh.org gfm
wdoc_1992G_r: /home/tj/gitclone/worg/org-contrib/org-watchdoc.org /home/tj/git/org-watchdoc/export-templates/org-watchdoc-worg.org org
wdoc_1992gas: /home/tj/git/org-watchdoc/targets/org-watchdoc.html /home/tj/git/org-watchdoc/export-templates/org-watchdoc-gh.org html
wdoc_1992tky: /home/tj/git/org-watchdoc/targets/org-watchdoc.txt /home/tj/git/org-watchdoc/export-templates/org-watchdoc-gh.org ascii
wdoc_1992fuB: /home/tj/git/org-watchdoc/targets/org-watchdoc.tex /home/tj/git/org-watchdoc/export-templates/org-watchdoc-gh.org latex
\end{verbatim}

Copyright (C) from 2014 Thorsten Jolitz
Author: Thorsten Jolitz <tjolitz at gmail dot com>
Keywords: org-mode, exporter, propagate changes, documentation

\subsection{MetaData}
\label{sec-1-1}
\begin{verbatim}
copyright: Thorsten Jolitz
copyright-years: 2014+
version: 1.0
licence: GPL 3 or later (free software)
licence-url: http://www.gnu.org/licenses/
part-of-emacs: no
git-repo: https://github.com/tj64/org-watchdoc.git
git-clone: git://github.com/tj64/org-watchdoc.git
author: Thorsten Jolitz
author_email: tjolitz AT gmail DOT com
\end{verbatim}
\subsection{Commentary}
\label{sec-1-2}

This library implements functionality for keeping exported files
associated with Org subtrees up-to-date.

Its principal use case is writing the comment-section of Emacs
Lisp (or other) source-code files only once (and in full Org-mode
using outorg.el), export it as README documentation from the
\textbf{outorg-edit-buffer} to html, ascii, latex/pdf,
markdown-github-flavor or whatever, and keep the exported doc
files automatically up-to-date when the original comment-section
of the source-buffer is edited explicitly with outorg (via M-x
outorg-edit-comments-and-propagate-changes).

org-watchdoc is just a little toolbox that can be used
independently from outorg too. All its functions are commands, so
its functionality is available for interactive use too.
\subsection{Installation}
\label{sec-1-3}

Put this line in your init file

\begin{verbatim}
(require 'org-watchdoc)
\end{verbatim}

and make sure Emacs can find the file org-watchdoc.el.

To take real advantage of org-watchdoc, outshine.el and outorg.el
(and maybe navi-mode.el) should be installed and source-code
buffers should be structured with outshine headers (outcommented
Org-mode headers), taking care that the whole comment-section is
one single outline tree that is the first headline in the
source-code file. Use org-watchdoc.el itself as an example for
this kind of file structuring.
\subsection{Usage}
\label{sec-1-4}

\subsubsection{Commands}
\label{sec-1-4-1}

Since org-watchdoc is a toolbox and not a mode, no menu or keymap
is specified. However, its commands can be used interactively:

\begin{center}
\begin{tabular}{ll}
M-x org-watchdoc- & action\\
\hline
add-target & add target-combination to watchlist\\
remove-target & remove target-combination from watchlist\\
propagate-changes & if md5 changed reexport all combinations\\
set-md5 & set org-watchdoc-md5 to current md5\\
\end{tabular}
\end{center}
\subsubsection{Interactive Use}
\label{sec-1-4-2}

In interactive use, this would be the typical order of actions:

\begin{enumerate}
\item Export first buffer tree to desired doc files (e.g. README-GH.md
or README-WORG.html). Optional, since adding non-exiting
target-files in step 2 will cause the exporter to write them the
when exiting the edit-buffer.

\item Add targets with point on first buffer headline.

Targets are combinations of files the exporter writes to,
export-template files to be inserted before the exporter does
its work, and backends the exporter should export to, e.g.

\begin{verbatim}
"/home/me/proj/README-GH.md /home/me/proj/gh-tmpl.org gfm"
"/home/me/proj/README-WORG.html /home/me/proj/worg-tmpl.org html"
\end{verbatim}

The three elements of such a combination are prompted from
the user.

\item Save and set md5 variable.

\item Edit the (narrowed) first buffer tree and save

\item Propagate changes.

Since md5 has changed due to the edits, all registered target
combinations are automatically re-exported.
\end{enumerate}
\subsubsection{Use with Outorg}
\label{sec-1-4-3}

Assuming outshine and outorg are installed and active, do once:

\begin{itemize}
\item Edit as Org

In the \textbf{outorg-edit-buffer} do steps 1 and 2 described above
for direct interactive use.
\end{itemize}

\begin{verbatim}
M-x outorg-edit-comments-and-propagate-changes
\end{verbatim}

Then whenever you want to edit the source-buffer's
comment-section and propagate the changes to the watched doc
files, do:

\begin{verbatim}
M-x outorg-edit-comments-and-propagate-changes
\end{verbatim}

instead of the usual 

\begin{verbatim}
M-x outorg-edit-comment-as-org
\end{verbatim}

This will

\begin{itemize}
\item Offer the first buffer tree for editing in the
\textbf{outorg-edit-buffer}

\item Reset `org-watchdoc-md5' immediately after edit-buffer setup

\item Check if buffer md5 has changed when editing is quitted. If so,
propagate the changes to the doc files registered in the subtrees
watchlist.
\end{itemize}
\subsubsection{Keybindings in Outshine}
\label{sec-1-4-4}

Here are the keybindings I added to outshine.el:

\begin{verbatim}
;; edit comment-section with `outorg' and propagate changes

;; best used with prefix-key 'C-c' 
(define-key map "`" 'outorg-edit-comments-and-propagate-changes)

;; best used with prefix-key 'M-#'
(define-key map "\M-+" 'outorg-edit-comments-and-propagate-changes)
(define-key map "+" 'outorg-edit-comments-and-propagate-changes)
\end{verbatim}

So just like editing e.g. an Emacs Lisp buffer or subtree (with
outshine activated) in full Org-mode only involves pressing M-\# M-\#
once to start editing, and then M-\# to exit the edit-buffer, edting
the comment-section of such a source-buffer and propagating the
changes to several export-targets only involves pressing M-\# M-+ once
to start editing, and then M-\# to exit the edit buffer (when M-\# was
set as outline-minor-mode prefix). 
\subsubsection{ChangeLog}
\label{sec-1-4-5}

\begin{center}
\begin{tabular}{llr}
date & author(s) & version\\
\hline
\textit{<2014-04-09 Mi>} & Thorsten Jolitz & 0.9\\
\end{tabular}
\end{center}
Emacs 24.3.1 (Org mode 8.2.5h)
\end{document}